\documentclass[a4paper,12pt]{article}
\usepackage{CJKutf8}

\begin{document}
\begin{CJK}{UTF8}{gbsn}

\section*{lserv2设计}
	本文档旨在lserv2的设计进行说明。(这其实是作业要求)

\section{编译和运行}
	这是针对linux编写的程序,在linux下执行
	\begin{verbatim}
make
	\end{verbatim}
	即编译好。然后执行
	\begin{verbatim}
./server2
	\end{verbatim}
	即在预定义的端口(2333)上开启了服务器。若要使用服务器,可用Netcat这个工具,即
	\begin{verbatim}
nc localhost 2333
	\end{verbatim}
	与该服务器进行互动。(lclie2.c也能互动)\newline
	具体可以使用的指令参见下文。

\section{参考以及原型}
	\subsection{参考书籍}
		《UNIX网络编程》
	\subsection{设计模式}
		TCP并发服务器,指令驱动

\section{设计}
	\subsection{实现概要}
		在实现了C/S的相互通信的基础上\footnote{也即,服务器和客户端可以在stdin中写入信息,
		然后发送给对方,对方在stdout中打印这条信息。如此实现互动。},在服务器端添加了功能,
		让服务器不仅能接收并打印收到的信息,还能解析客户端发送的信息。\newline
		lserv2采用的模式是
		\begin{description}
		\item[客户端] 客户端只负责发送和接收信息,剩下的所有处理过程都在服务器端发生。
		用户使用客户端的时候,用户输入指令,指令发送到服务器,服务器解析并进行相应反馈。
		\item[服务器] 接收指令,解析,反馈。具体功能参见下文。
		\end{description}
	\subsection{具体功能}
		以下命令都是




\end{CJK}
\end{document}
